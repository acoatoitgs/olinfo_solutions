%This template is a customization of the CTM template:
%https://github.com/SeniorMars/dotfiles/tree/master/latex_template
%You may use this template as you wish, crediting is optional.
\documentclass{report}
\usepackage{hyperref}
\usepackage{fancyvrb}
\usepackage{fancyhdr}
\usepackage{listings}
\usepackage{enumitem}
\usepackage{accsupp}


\input{preamble}
\input{macros}
\input{letterfonts}
%--------------------Setup--------------------
 
\definecolor{commentcolor}{RGB}{119,118,123}
\definecolor{keywordscolor}{RGB}{152,102,255}
\definecolor{stringcolor}{RGB}{0,153,51}
\definecolor{backcolor}{rgb}{0.92,0.92,1}

\definecolor{lightlightblue}{rgb}{0.92,0.92,1}
\definecolor{lightblue}{rgb}{0.7,0.7,1}

\lstdefinestyle{mystyle}{
    backgroundcolor=\color{backcolor},   
    commentstyle=\color{commentcolor},
    keywordstyle=\color{keywordscolor},
    stringstyle=\color{stringcolor},
    basicstyle=\ttfamily\footnotesize,
    breakatwhitespace=false,         
    breaklines=true,                 
    captionpos=b,                    
    keepspaces=true,                 
    numbers=none,                    
    numbersep=5pt,                  
    showspaces=false,                
    showstringspaces=false,
    showtabs=false,                  
    tabsize=2,
    columns=fullflexible
}
\lstset{style=mystyle}
\pagestyle{fancy}
\lhead{Andrea Coato}
\rhead{Last edited on \today}

\hypersetup{
    colorlinks=true,
    urlcolor=blue,
}

\newcommand{\code}[1]{\texttt{#1}}
%######################################################
%######################################################
%--------------------Actual document-------------------
%######################################################
%######################################################


\begin{document}
\section*{Trova il massimo ($easy1$)}
Dato un array di interi, il problema chiede di trovare il massimo valore in esso.

\begin{tcolorbox}[colback=lightlightblue,colframe=lightblue, , coltitle=black, title=Soluzione in $\mathcal{O}(N)$]
    Utilizziamo la funzione \code{std::max\_element} che ha complessità $\mathcal{O}(N)$.
\end{tcolorbox}

\begin{lstlisting}[language=c++]
    #include <bits/stdc++.h>
    using namespace std;
    
    int n;
    int main() {
    
          freopen("output.txt", "w", stdout);
    
        std::ifstream file;
    
        file.open("input.txt");
    
        file >> n;
        std::string str;
        std::vector<int> v(n);
        
    
        for(int i = 0;std::getline(file, str, ' '); i++ )
        {
            v[i] = std::stoi(str);
        }
        
    
        int t = *max_element(v.begin(), v.end());
    
        std::cout << t;
        return 0;
    }
\end{lstlisting}

\end{document}